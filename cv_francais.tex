\documentclass[10pt,a4paper]{moderncv}
\moderncvtheme[blue]{classic}
\usepackage[utf8]{inputenc}
%\usepackage[scale=0.8]{geometry}
\usepackage[top=1.1cm, bottom=1.1cm, left=2cm, right=2cm]{geometry}

\firstname{Christophe}
\familyname{Cluizel}
\title{Étudiant Ingénieur \newline \newline \mbox{en Architecture} \newline \newline \mbox{des Systèmes d'Information}}
\quote{Stage - Data science \\ À partir de février (6 mois)}
\address{Résidence Wallon Appt.14}{28 rue du Madrillet}{76800 Saint-Étienne-du-Rouvray}
\mobile{07.86.13.54.96}
\email{christophe.cluizel@gmail.com}
\social[twitter]{CCluizel}
\social[github]{ChristopheCluizel}
\extrainfo{23 ans - Permis B}
% \photo[64pt][0.4pt]{photo}

\begin{document}
  \maketitle

  % ========= Diplômes et Études ========
  \section{Diplômes et Études}

    \cventry{2011 - 2016}{INSA}{Architecture des Systèmes d'Information}{}{Informatique et système d’information (acquisition, traitements et restitution de l’information), data-mining, machine learning (SVM, clustering, regression, NNs, random-forest), théorie des graphes}{Institut National des Sciences Appliquées -- Rouen(76)}

    \cventry{2010 - 2011}{Classe préparatoire MPSI}{Mathématiques -- Physique -- Sciences de l’Ingénieur}{}{}{Lycée Corneille -- Rouen(76)}


  % ========= Projets et Stages ========
  \section{Projets et Stages}

    \cventry{Juin 2015 \\ (3 mois)}{Avari}{Data-mining}{Berlin}{Allemagne}{
    \begin{itemize}
      \item \emph{\textbf{Stage data scientist}}
      \item \emph{Optimisation d'un système de recommandation implicite pour site de e-commerce}
      \item \emph{Visualisation de données avec Librato}
      \item \emph{Analyse de données à l'aide de scripts Python}
    \end{itemize}}

    \cventry{Janvier 2015 \\ (1 an)}{Orange Vallée}{Data-mining}{}{France}{
    \begin{itemize}
      \item \emph{\textbf{Développeur} (1 an) - \textbf{Chef de projet} (6 mois)}
      \item \emph{Data-mining et étude de réseaux sociaux}
      \item \emph{Étude de données et analyse de graphe à l'aide de Spark et GraphX en Scala}
      \item \emph{Visualisation de données avec Zeppelin}
    \end{itemize}}

    \cventry{Octobre 2014 \\ (1 mois et demi)}{SGA Automation}{Borne de rechargement-location automobile}{Rouen(76)}{France}{
    \begin{itemize}
      \item \emph{\textbf{Chef de projet} - \textbf{Développeur}}
      \item \emph{Réalisation d'un système de communication entre des bornes autonomes et un serveur}
    \end{itemize}}

    \cventry{Juillet 2012 \\ (1 mois)}{Acim Jouanin}{Conception de résistances}{Évreux(27)}{France}{
    \begin{itemize}
      \item \emph{\textbf{Opérateur} sur tous les postes de production}
    \end{itemize}}


  % ========= Informatique ========
  \section{Informatique}

    \cvcomputer{\textbf{Langage}}{Scala, Python, \LaTeX, Matlab, Java, C++}{\textbf{Outil}}{IntelliJ, Sublime Text, Git, Eclipse}
    \cvcomputer{\textbf{Bibliothèque}}{Spark/GraphX, Scipy, OpenCV}{\textbf{OS}}{GNU/Linux, Windows}
    \cvcomputer{\textbf{Base de données}}{PostgreSQL, MySQL}{\textbf{Logiciel}}{Zeppelin, ElasticSearch, Kibana, Librato, S3, EC2}


  % ========= Langues ========
  \section{Langues}

    \cvcomputer{\textbf{Français}}{Langue maternelle}{\textbf{Allemand}}{Bon niveau}
    \cvcomputer{\textbf{Anglais}}{Courant (TOEIC 900)}{\textbf{Espagnol}}{Notions}


  % ========= Centres d'intérêt et Loisirs ========
  \section{Centres d'intérêt et Loisirs}

    \cvline{Associatif}{- Adhérent à l'Association de \textbf{Robotique} de l’INSA -- 3 ans (\textbf{président} -- 2 ans), \newline participation à la Coupe de France de Robotique en équipe -- 3 ans.}
    \cvline{}{- Membre de l'\textbf{AJIR} (Association Junior Insa Rouen) -- 6 mois, chef de projet}
    \cvline{Musique}{\textbf{Piano} -- 5 ans (cours particulier)}
    \cvline{Sport}{\textbf{Tennis de table} -- 7 ans (compétitions départementales individuelles et par équipe)}
    \cvline{MOOC}{Sur Coursera : \textbf{Machine Learning} (Stanford)}
    \cvline{Voyage}{Nombreux \textbf{séjours à l'étranger} : Angleterre, Allemagne, Espagne, République tchèque, Maroc.}
\end{document}
